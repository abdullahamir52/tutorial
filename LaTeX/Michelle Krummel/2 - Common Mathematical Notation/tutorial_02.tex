\documentclass[11pt]{article}
\pagestyle{empty}
\usepackage{amsmath, amssymb, amsfonts }

\begin{document}

\section{Superscript}

First we will learn about superscripts.

We need to use dollar sign before we write a mathematical notation. Otherwise it will give an error. 

$2x^3$

Two dollar symbol will generate a separate line for the math formula. 

$$ 2x^34 $$

Use curly brackets to enclose the power/exponent component in your mathematical
formula. Otherwise, latex will only count immediate number after the power
symbol. 

$$ 2x^{34} $$

$$ 2x^{3x+4} $$

$$ 2x^{3x^4 +5}  $$

\section{Subscript}

'slash ldots' gives us three dots.
'slash cdots' gives us three dots, aligned at the center.


$$ x_1 $$

$$ x_{12} $$

$$ x_{1_2} $$

$$ x_{1_{2_3}} $$

$$ a_0, a_1, a_2, \ldots, a_{100} $$

\section{Greek letters}

$$ \pi $$

$$ \Pi $$

$$ \alpha $$ 

$$ A = \pi r^2 $$

We need to put a space in between Greek letter command, and later commands. 

\section{Trig Functions}

$$ y = sinx $$

Here, everything is italicized. We just want the angle to be italicized. To do that, put a slash in front of sin. 

$$ \sin x$$

$$ \cos x$$

$$ \csc \theta$$

$$ \csc \Theta$$

$$ y= \sin^{-1} \theta $$

$$ y=\arcsin x $$

\section{ log functions}

$$y= logx$$

$$ y = \log x $$

$$ y= \log_5 x $$

$$ y = \ln x $$

\section{roots}

$$ \sqrt{2}$$

$$ \sqrt[3]{4} $$

$$ \sqrt{x^2 + y^2}$$

$$ \sqrt { 1	+  \sqrt{x}	} $$

\section{fractions}

$$ \frac{2	}	{3} $$

We use two types of fraction writing style: one is small, the other one is large. 
\\ [6pt]
About $\frac{2}{3}$ of the glass is full.
\\ [6pt]
About $\displaystyle \frac{2}{3}$ of the glass is full. 
\\ [6pt]
About $\dfrac{2}{3}$ of the glass is full.
\\ [6pt]

$$	 \frac{\sqrt{x+1}}{\sqrt{x+2}}  $$

$$ \frac  {1}   {  1 +  \frac  {1}  {x}  } $$

\end{document}