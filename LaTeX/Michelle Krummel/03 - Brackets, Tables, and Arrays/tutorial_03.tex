\documentclass[11pt]{article}
\usepackage[english]{babel}
\usepackage{amsfonts, amssymb, amsmath}
\usepackage{float}
\parindent 0px
\pagestyle{empty}


\usepackage{multirow}

\usepackage{graphicx}

\usepackage{dcolumn}

\usepackage[table]{xcolor}

\usepackage{tabularx}

\usepackage{array}




%here we're setting up a version of the math fonts with normal x-width
\DeclareMathVersion{nxbold} 
\SetSymbolFont{operators}{nxbold}{OT1}{cmr} {b}{n}
\SetSymbolFont{letters}  {nxbold}{OML}{cmm} {b}{it}
\SetSymbolFont{symbols}  {nxbold}{OMS}{cmsy}{b}{n}

\begin{document}

\section{Brackets}

The distributive property states that $ a(b+c) = ab + ac$, for all $a, b, c \in \mathbb{R} $.\\[6pt]
The equivalence class of $a$ is $[a]$\\[6pt]
The set $A$ is defined to be ${1,2,3}$. Here the curly brackets are not shown. Curly bracket is a reserved word. So if you want to show the curly brackets, just type a slash in front of it. Like $\{1,2,3\}$\\[6pt]
The movie ticket costs $\$11.50$. 
\\[6pt]
$2(\frac{1}{x^2-1})$\\[6pt]
$$2\left(\frac{1}{x^2-1}\right)$$\\[6pt]
$$2\left\{\frac{1}{x^2-1}\right\}$$\\[6pt]
$$2\left[\frac{1}{x^2-1}\right]$$\\[6pt]
$$2\left \langle   \frac{1}{x^2-1}    \right \rangle    $$\\[6pt]
$$2\left |   \frac{1}{x^2-1}    \right |    $$\\[6pt]
$$ \left. \frac{dy}{dx} \right| _{x=1}  $$
$$ \left(   \frac{1} { 1 + \left( \frac{1}{1+x}  \right) }  \right)   $$

\section{Tables}


\begin{tabular}{|c||c|c|c|c|c|}
\hline
$x$ & 1 & 2 & 3 & 4 & 5 \\ \hline
$f(x)$ & 10 & 11 & 12 & 13 & 14 \\ \hline
\end{tabular}

\vspace{1cm}




\begin{table} [H]
\def\arraystretch {1.5}
\centering

\begin{tabular}{|c||c|c|c|c|c|}
\hline
$x$ & 1 & 2 & 3 & 4 & 5 \\ \hline
$f(x)$ & $\frac{1}{2}$ & 11 & 12 & 13 & 14 \\ \hline
\end{tabular}

\caption{These values represent the function $f(x)$}
\end{table}




\begin{table} [H]
\def\arraystretch {1.5}
\centering

\caption{The relationship between and $f$ and $f'$}
\begin{tabular}{|c|c|}
\hline
$f(x)$ & $f'(x)$ \\ \hline 
$ x >0 $  & The function $ f(x) $ is increasing \\ \hline
\end{tabular}

\end{table}





\begin{table} [H]
\def\arraystretch {1.5}
\centering

\caption{The relationship between and $f$ and $f'$}
\begin{tabular}{|l|p{12cm}|}
\hline
$f(x)$ & $f'(x)$ \\ \hline 
$ x >0 $  & The function $ f(x) $ is increasing. The function $ f(x) $ is increasing. The function $ f(x) $ is increasing. The function $ f(x) $ is increasing. The function $ f(x) $ is increasing.  \\ \hline
\end{tabular}

\end{table}


\section{Equation arrays}

\begin{align}
5                    x^2 \text{ place your words here.}
\end{align}

\begin{align}
5                    x^2      \,     \text{place your words here.}
\end{align}


\begin{align*}
5x^2-9     &=    x+3 \\
5x^2 -x -12    &=    0 \\
&= 12+x -5x^2
\end{align*}


\begin{align}
5x^2-9     &=    x+3 \\
5x^2 -x -12    &=    0 \\
&= 12+x -5x^2
\end{align}



\end{document}