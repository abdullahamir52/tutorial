\documentclass[12pt,a4paper]{article}
\usepackage[utf8]{inputenc}
\usepackage[english]{babel}
\usepackage{amsmath}
\usepackage{amsfonts}
\usepackage{amssymb}
\usepackage{graphicx}
\usepackage{lmodern}
\usepackage[left=2cm,right=2cm,top=2cm,bottom=2cm]{geometry}
\author{Chowdhury Amir Abdullah}

\usepackage{float}

\begin{document}


The rain in Spain falls \emph{mainly} on the plain.


\begin{itemize}
\item Tea
\item Milk
\item Biscuits
\end{itemize}



\begin{figure}[H]
\includegraphics{1}
\end{figure}


\begin{equation}
\alpha + \beta + 1
\end{equation}


% For the most part, you can just type your text normally.

Words are separated by one or more
spaces.

Paragraphs are separated by one
or more blank lines.


% Space in the source file is collapsed in the output.
The rain			 in 					Spain
falls mainly 				on the				 plain.



Quotation marks are a bit tricky:
use a backtick on the left and an apostrophe on the right.

Single quotes: `text' . 

Double quotes: ``text'' . \\ [6pt]


Some common characters have special meanings in \LaTeX\: \\
\$\%\&\#!


In March 2006, Congress raised that ceiling an additional $\$0.79$
trillion to $ \$8.97$ trillion, which is approximately $68\%$ of GDP.  As of
October 4, 2008, the ``Emergency Economic Stabilization Act of
2008'' raised the current debt ceiling to $\$11.3$ trillion. \\[6pt]

In March 2006, Congress raised that ceiling an additional \$0.79 trillion to \$8.97 trillion, which is approximately 68\% of GDP. As of October 4, 2008, the ``Emergency Economic Stabilization Act of 2008'' raised the current debt ceiling to \$11.3 trillion.\\[6pt]


% not so good:
Let a and b be distinct positive
integers, and let c = a - b + 1. \\[6pt]

% much better:
Let $a$ and $b$ be distinct positive
integers, and let $c = a - b + 1$. \\[6pt]

% LATEX handles spacing automatically; it ignores your spaces.
Let $y=mx+b$ be \ldots\\[6pt]

Let $y = m x + b$ be \ldots\\[6pt]

 Use caret  for superscripts and underscore  for subscripts.\\[6pt]
$y = c_2 x^2 + c_1 x + c_0$ \\[6pt]

 Use curly braces to group superscripts and subscripts.\\[6pt]
$F_n = F_n-1 + F_n-2$ \\[6pt] % oops!
$F_n = F_{n-1} + F_{n-2}$ \\[6pt] % ok!

There are commands for Greek letters and common notation. \\[6pt]
$$\mu = A e^{Q/RT}$$\\[6pt]
$$\Omega = \sum_{k=1}^{n} \omega_k$$\\[6pt]

% If it's big and scary, display it on its own line using\begin{equation} and \end{equation}.

The roots of a quadratic equation are given by: 
\begin{equation}
x = \frac{-b \pm \sqrt{b^2 - 4ac}}
{2a}
\end{equation}
where $a$, $b$ and $c$ are \ldots the coefficients of the quadratic equation. \\[6pt]

Caution: LATEX mostly ignores your spaces in mathematics, but it can't handle blank lines in equations | don't put blank lines in your mathematics.\\[6pt]

% Note how the sigma is bigger in the equation environment, and how the subscripts and superscripts change position, even though we used the same commands.
% In fact, we could have written $...$ as \begin{math}...\end{math}.

\newpage

\begin{itemize} % for bullet points
\item Biscuits
\item Tea
\end{itemize}


\begin{enumerate} % for numbers
\item Biscuits
\item Tea
\end{enumerate}




%All of the commands and environments we've used so far are built into LATEX.

%Packages are libraries of extra commands and environments.

%There are thousands of freely available packages.

%I We have to load each of the packages we want to use with a \usepackage command in the preamble.
% Example: amsmath from the American Mathematical Society.

%Use equation* (\equation-star") for unnumbered equations.

\begin{equation*}
\Omega = \sum_{k=1}^{n} \omega_k
\end{equation*}


%LATEX treats adjacent letters as variables multiplied together, which is not always what you want. amsmath denes commands for many common mathematical operators.

\begin{equation*} % bad!
min_{x,y} (1-x)^2 + 100(y-x^2)^2
\end{equation*}


\begin{equation*} % good!
\min_{x,y}{(1-x)^2 + 100(y-x^2)^2}
\end{equation*}


%You can use \operatorname for others.

\begin{equation*}
\beta_i =
\frac{\operatorname{Cov}(R_i, R_m)}
{\operatorname{Var}(R_m)}
\end{equation*}


%Align a sequence of equations at the equals sign with the align* environment.

\begin{align*}
(x+1)^3 &= (x+1)(x+1)(x+1) \\
&= (x+1)(x^2 + 2x + 1) \\
&= x^3 + 3x^2 + 3x + 1
\end{align*}


An ampersand \& separates the left column (before the =) from the right column (after the =). \\[6pt]
 A double backslash \ \ starts a new line.\\[6pt]

Let $ X_1, X_2,  \ldots, X_n $ be a sequence of independent and identically distributed random variables with $ E[X_i] = \mu $ and VAR$[X_i] = \sigma ^2 < \infty $, and let

\begin{equation*}
S_n = \frac{1}{n} \sum_{i=1}^{n} X_i
\end{equation*}

denote their mean. Then as $n$ approaches infinity, the random variables $ \sqrt{n} (S_n - \mu) $ converge in distribution to a normal $ N(0, \sigma ^2)$ \\[6pt]


Let $X_1, X_2, \ldots, X_n$ be a sequence of independent and identically distributed random variables with
$\operatorname{E}[X_i] = \mu$ and
$\operatorname{Var}[X_i] = \sigma^2 < \infty$, and let
\begin{equation*}
S_n = \frac{1}{n}\sum_{i=1}^{n} X_i
\end{equation*}
denote their mean. Then as $n$ approaches infinity, the random variables $\sqrt{n}(S_n - \mu)$ converge in distribution to a normal $N(0, \sigma^2)$. $\mathcal{N}(0, \sigma^2)$ \\[6pt]

% bonus points: the N for normal is usually set in a caligraphic font; you can get this using $\mathcal{N}(0, \sigma^2)$.

\end{document}